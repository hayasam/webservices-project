\subsection{Introduction to Web services }
What exactly is a web service? It is a software function, that supports machine-to-machine interactions over a network. Mostly web services are used for connecting existing software stubs into one usable service, but there is a broad variety of web services usage. With the help of web services we can solve the interoperability problem when linking or exchanging data. Web services describe a way of integrating web-based applications using common standards like XML, SOAP, WSDL and UDDI. It is also broadly used for business to communicate with each other and with clients. It does not provide a graphical user interface for the user, but it shares business logic, data and processes through a programmatic interface, therefore, the usage of Web Services has a very important role in business processing.

Web Services Discovery - in order to make a web service usable, service provider needs to make it accessible to potential users. It can be located through a public or private business registry, or a WSIL document. Discovery is managed by UDDI (Universal Description Discovery and Integration), where distributed registry of business and its services implementation is being set for potential clients. WSIL is an alternative to UDDI, which allows to go directly to the web service and ask for the possible services to use. Discovery works as the client is looking for one or more services and the services, and then as a result, which match the criteria is being sent as a response directly to the client. After discovery process completion, the service provider and client application is able to know the exact location of service (URI), its capabilities and how to interact with it.

Web Service Composition is being more and more used, because composite services have unique features, that an individual service cannot support. Semantic web is a more advanced form of the web, where all the contents have well defined meaning, because of composition. In web service composition services can automatically select, integrate and invoke different other web services in order to achieve user specified tasks. Composing existing web services to deliver new functionality should be used by all the companies, because it provides new functionality with the ease of use. For example, LameDuck, NiceView and TravelGood together was made as a composition, when planning, booking and payment connects with each other and is being used as one dependent application. Composition extends the notion of service discovery by enabling automatic composition of services to meet the requirements of a given task description. It can be either static (where requirements are pre-defined) or dynamic (requirements are given on the spot). It connects a variety of different services in a composite application, so many services can be used in one application and one web service is dependent on another. 

WS-Coordination is a Web Services specification which defines a framework for protocols that coordinate the actions of distributed applications. Coordination is one of the major parts in business processing. Such coordination protocols are used to support a number of applications, including those that need to reach a consistent agreement on the outcome of distributed transactions. The framework defined in this specification enables an application service to create a context needed to propagate an activity to other services and to register for coordination protocols. 

RESTful Services (Representational state transfer) is an abstraction of the architecture of the web. It is an architectural style consisting of a coordinated set of architectural constraints applied to components, data elements within a distributed system. If applied to a web service, REST improves desirable properties such as performance, scalability on a distributed system. Data, functionality and resources are accessed using Uniform Resource Identifiers. REST manipulates resources using a fixed set of operations: PUT, GET, POST and DELETE, where PUT creates a new resource, which can be deleted by DELETE, GET retrieves the current state of the resource and POST commits a new resource state. 

Service orientation is a design paradigm to build computer software in the form of services. The idea behind this is to automate the business logic as a distributed system. System orientation is a set of design principles which carries out the separation of concerns in the software. Orientation promotes loose coupling between software components so that it can be reused. The idea of service orientation is to do a service that is easy to use. 

Basic service technologies have components such as XML, SOAP, WSDL, UDDI. First of all, web services communicates using SOAP messages. They are being exchanged using XML style documents. There are also two types of SOAP interaction styles: document and RPC style. The main difference between RPC and document style is that RPC style pattern is used, when the client uses web service as a single application with data inside. Request and response messages map directly to the input and output parameters. For example single input for converting temperature from celsius to fahrenheit, contains simple input and output. However when document style is used, client uses web service as a long running business process. Here input parameters are a complete unit of information (using an instance of another web service as an element), such as transaction, where credit card information must be verified during the process. And there main web service calls another web services to use another service helping functionality.

WSDL (Web Service Description Language) is an XML based document which contains a set of definitions that describe the functionality offered by a web service. It has its own structure where types, messages, operations, port types, bindings and services are defined. That document defines services as collections of networks endpoints, or ports. In WSDL, it is possible and important to be able to describe the communication between devices using web services in a structured manner. 

XML (eXtensible Markup Language) is a markup language that defines a set of rules for encoding documents which is readable for both: machines and people developing a code. The goals of XML emphasize simplicity, generality and usability across the Internet, but the main usage of XML in web services is message exchanging in SOAP. 

